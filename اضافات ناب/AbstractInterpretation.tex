



\section{نظریه تعبیر مجرد}

به‌طور خلاصه، نظریه تعبیر مجرد یک چارچوب برای ساختن یک تقریب از معناشناسی یک زبان‌ برنامه نویسی است. 

معناشناسی یک زبان یک مدل ریاضیاتی مجرد است که چگونگی رفتار برنامه‌ها در این زبان را توصیف می‌کند. 
تقریب نیز یک معناشناسی دیگر است که قرار است بخشی( نه همه) از رفتارهای یک برنامه‌ی کامپیوتری در حال اجرا در یک زبان را توصیف کند. این که تقریب چیست، یک معناشناسی را در چه زمانی می‌توانیم تقریبی برای معناشناسی دیگری بدانیم و از یک تقریب چه چیزهایی را می‌توانیم بفهمیم و مواردی دیگر در مورد ارتباط بین دو مدل ریاضیاتی که درباره‌ی معنای برنامه‌ها در یک زبان برنامه‌نویسی واحد صحبت می‌کنند، همگی موضوع بحث در نظریه‌ي تعبیر مجرد است.
پس تا اینجا مشخص شد که نظریه‌ی تعبیر مجرد در مورد ارتباط بین معناشناسی‌های مختلف صحبت می‌کند. 

برای شروع بحث صوری در مورد این نظریه، از مفهوم دامنه و معناشناسی شروع می‌کنیم.
در واقع، این نوع از مشخص کردن معناشناسی یک زبان برنامه نویسی را معناشناسی دِلالَتی نامیده‌اند. در فصول آینده با یک معناشناسی از این نوع سر و کار خواهیم داشت.
\begin{defn}
	(معناشناسی و دامنه): اگر $\mathbb{P}$ مجموعه‌ی برنامه‌ها در یک زبان برنامه نویسی باشد، به تابع 
	$\mathcal{S}:\mathbb{P} \rightarrow D$
	یک معناشناسی و به مجموعه‌ی $D$ یک دامنه می‌گوییم.
\end{defn}
همان‌طور که از تعریف مشخص است، برای این که بتوانیم معنای برنامه‌های کامپیوتری موجود در یک زبان را تعریف کنیم، به یک مجموعه به اسم دامنه احتیاج داریم. تلاش برای پی بردن به این که در یک معناشناسی باید چه مجموعه‌ای را به عنوان دامنه در نظر گرفت، منجر به تولد یک مبحث به نام نظریه‌ی دامنه شده است.

در فصل‌های بعدی، با یک معناشناسی دلالتی سر و کار خواهیم داشت. 
پس از ارائه‌ی یک زبان برنامه نویسی، یک معناشناسی برای آن زبان معرفی می‌کنیم که معناشناسی رد پیشوندی نام دارد. در این معناشناسی، دامنه یک مجموعه است که شامل موجوداتی به نام رد پیشوندی است. هر رد پیشوندی یک دنباله است که در هر عضو آن اطلاعات موجود در حافظه و مرحله‌ی اجرای برنامه مشخص شده است. 

اما فعلا که در حال صحبت در مورد نظریه‌ی تعبیر مجرد هستیم، معناشناسی خاصی را معرفی نمی‌کنیم و بحث را کلی‌تر پیش می‌بریم. نظریه تعبیر مجرد برای معناشناسی‌ها یک چارچوب مشخص کرده و فقط در مورد معناشناسی‌هایی که در این چارچوب می‌گنجند می‌تواند صحبت کند. یکی از محدودیت‌های این چارچوب این است که دامنه باید یک ترتیب جزئی باشد.
\begin{defn}
	(ترتیب جزئی): یک مجموعه‌ی $D$ را به همراه یک رابطه‌ی $\leq$ روی آن مجموعه ترتیب جزئی می‌گوییم، اگر و تنها اگر خواص زیر را داشته باشند:
	$$\blacktriangleleft \forall a \in D: a\leq a$$
	$$\blacktriangleleft \forall a,b \in D: a \leq b \land b \leq a \rightarrow a=b$$
	$$\blacktriangleleft \forall a,b,c \in D: a \leq b \land b \leq c \rightarrow a \leq c$$
\end{defn}

حال به تعریف بخش بزرگتری از این چارچوب می‌پردازیم. در جبر مجرد مفهومی به اسم تناظر گالوا وجود دارد. این تناظر بین مجموعه‌ای از گروه‌ها و مجموعه‌ای از توسیع میدان‌هایی خاص وجود دارد که به بحث ما مربوط نمی‌شوند. این تناظر یک شکل نظریه ترتیبی هم دارد که در آن به جای مجموعه‌ای از گروه‌ها و میدان‌ها، دو مجموعه‌ی جزئا مرتب داریم. می‌توان گفت در واقع این یک مجرد سازی تناظری است که از جبر آمده. 

به شکل ضعیف‌تر نظریه ترتیبی این تناظر اتصال گالوا می‌گویند که در نظریه تعبیر مجرد به عنوان شرط تقریب تعریف شده است، به این معنی که دامنه‌ی یک معناشناسی باید با دامنه‌ی تقریبش یک اتصال گالوا داشته باشد.

\begin{defn}
	(اتصال گالوا): برای دو ترتیب جزئی
	$(A,\leq)$ 
	و
	$(C,\subseteq)$
	زوج 
	$\langle \alpha , \gamma \rangle$
	شامل دو تابع
	$\gamma:A \rightarrow C$
	و
	$\alpha:C \rightarrow A$،
	یک اتصال گالوا است اگر و تنها اگر
	$$\forall c \in C :\forall a \in A: \alpha(c)\leq a \leftrightarrow c \subseteq \gamma(a)$$.
\end{defn}









