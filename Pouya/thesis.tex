%%%%%%%%%%%%%%%%%%%%%%%%%%%%%%%%%%%%%%%%%
% Focus Beamer Presentation
% LaTeX Template
% Version 1.0 (8/8/18)
%
% This template has been downloaded from:
% http://www.LaTeXTemplates.com
%
% Original author:
% Pasquale Africa (https://github.com/elauksap/focus-beamertheme) with modifications by 
% Vel (vel@LaTeXTemplates.com)
%
% Template license:
% GNU GPL v3.0 License
%
% Important note:
% The bibliography/references need to be compiled with bibtex.
%
%%%%%%%%%%%%%%%%%%%%%%%%%%%%%%%%%%%%%%%%%

%----------------------------------------------------------------------------------------
%	PACKAGES AND OTHER DOCUMENT CONFIGURATIONS
%----------------------------------------------------------------------------------------

\documentclass{beamer}

\usetheme{focus} % Use the Focus theme supplied with the template
% Add option [numbering=none] to disable the footer progress bar
% Add option [numbering=fullbar] to show the footer progress bar as always full with a slide countx


\definecolor{main}{RGB}{55, 135, 177}
\definecolor{text}{RGB}{60, 60, 80}
\definecolor{background}{RGB}{255, 255, 255}

\setbeamercovered{transparent}

%------------------------------------------------

\usepackage{booktabs} % Required for better table rules
\usepackage[utf8]{inputenc}
\usepackage[T1]{fontenc}
\usepackage{amsmath}
\usepackage{amsfonts}
\usepackage{amssymb}
\usepackage{amsthm}
\usepackage{mathrsfs}
\usepackage{tikz}
\usepackage[version=4]{mhchem}
\usepackage{stmaryrd}
\usepackage{bbold}
\usepackage{mathrsfs}
\usepackage{CJKutf8}
\usepackage{xepersian}
\settextfont{Vazir}
\setsansfont{FreeSerif}
\renewcommand{\baselinestretch}{1.3}


\title{\rl{بهبود روش وارسی مدل با استفاده از تعبیر مجرد}}

\subtitle{\rl{پویا پرتو}}

\author{\rl{دکتر مجید علی‌زاده و دکتر مجتبی محتهدی}}

\titlegraphic{\includegraphics[scale=.21]{images/tu-logo-hckd.png}}

\institute{\rl{	دانشکده‌ی ریاضی، آمار و علومِ کامپیوتر \\ دانشگاهِ تهران}}

\date{\rl{شهریور ۱۴۰2}}




\begin{document}


\begin{frame}
	\maketitle
\end{frame}
%--------------------------------
%پیشگفتار

\section{\rl{پیشگفتار}}
\begin{frame}{\rl{درستی‌یابی برنامه‌های کامپیوتری}}
	\begin{flushright}
	\rl{$\bullet$
		اهمیت درستی‌یابی برنامه‌ها\\
		\vspace{1em}
	\pause
	$\bullet$
		روش‌های پویا\\
	\vspace{1em}
	\pause
	$\bullet$
	روش‌های ایستا
}	
	\end{flushright}
\end{frame}


\begin{frame}{\rl{روش‌های ایستا}}
	\begin{flushright}
		\rl{$\bullet$
			روش‌های صوری: استفاده از نظریه‌های مختلف در منطق ریاضی\\
			\vspace{1em}
			\pause
			$\bullet$
			قضیه‌ی رایس: تصمیم ناپذیری مسئله در حالت کلی\\
			\vspace{1em}
			\pause
			$\bullet$
			روش وارسی مدل: استفاده از منطق‌های وجهی مختلف در کنار مدل کردن برنامه با کمک معناشناسی منطق\\
			\vspace{1em}
			\pause
			$\bullet$
			روش‌های استنتاجی: استفاده از نظریه نوع یا منطق تفکیک (منطق هور) و اثبات‌یار‌ها (مانند $\mathsf{Coq,Isabelle}$ و ...)\\
			\vspace{1em}
			\pause
			$\bullet$
			استفاده از نظریه تعبیر مجرد: تخمین زدن 
			\textbf{درست}
			 معناشناسی برنامه‌های کامپیوتری 
		}	
	\end{flushright}
	
	
\end{frame}

\begin{frame}{\rl{هدف}}
	\begin{flushright}
		\rl{$\bullet$
			بیان دوباره روش وارسی مدل با استفاده از عبارات منظم به جای منطق‌های وجهی\\
			\vspace{1em}
			\pause
			$\bullet$
			فایده: آشنا بودن برنامه نویسان با عبارات منظم\\
			\vspace{1em}
			\pause
			$\bullet$
			روش‌ وارسی مدل در سه صورت جدید بیان می‌شود.\\
		}	
	\end{flushright}
\end{frame}
%----------------------------------
\section{\rl{مفاهیم اولیه}}

\begin{frame}{\rl{منطق زمانی خطی}}
\begin{flushright}
	\begin{block}{\rl{نحو}}
		\vspace{-0.75 cm}
		\begin{center}
		$$
		\phi \in \Phi \Leftrightarrow
		\phi ::= \pi | \phi \lor \phi |
		\neg \phi |
		\bigcirc \phi |
		\phi \mathcal{U}\phi \;\;\;\;\;(\pi \in \Pi) 
		$$	
		\end{center}
	\end{block}	
\pause
	\begin{block}{\rl{معناشناسی}}
		\vspace{-0.5 cm}
		\begin{center}
		$	M,i \models \pi \:\:\: \mathit{iff} \:\:\: \pi \in M(i),$\\
		$	M,i \models \neg \phi \:\:\: \mathit{iff} \:\:\: M,i\nvDash \phi,$\\
		$	M,i \models \phi \lor \psi \:\:\: \mathit{iff} \:\:\: M,i \models \phi \:\:\: \mathit{or} \:\:\: M,i \models \psi,$\\
		$M,i \models \bigcirc \phi  \:\:\:  \mathit{iff} \:\:\: M,i+1 \models \phi,$\\
		$M,i \models \phi \mathcal{U} \psi \:\:\: \mathit{iff} \:\:\: 
		\exists k \geq i \in \mathbb{N}_0: \forall i\leq j< k: M,j \models \phi \:\:\: \mathit{and} \:\:\: M,k \models \psi.$
		\end{center}	
\end{block}
\end{flushright}
\end{frame}

\begin{frame}{\rl{نحو زبان مورد بررسی}}

		\begin{flushleft}
			\setlength{\parindent}{3.5cm}
			\begin{small}
			\lr{	
			\hspace{4.5cm}$\mathsf{x,y},... \in \mathbb{X},$\\
			$\mathsf{A} \in \mathbb{A} ::=\mathsf{1\:|\:x\:|\:A_1 - A_2}$\\
			$\mathsf{B \in \mathbb{B} ::=A_1<A_2 \:|\: B_1 \: nand\: B_2}$\\
			$\mathsf{E \in \mathbb{E}::= A \: | \: B}$\\
			$\mathsf{S\in \mathbb{S} ::=  }$\\
			$\hspace{1.5cm}\mathsf{x\doteq A; }$\\
			$\hspace{1.5cm}\mathsf{|\:\:\:;}$\\
			$\hspace{1.5cm}\mathsf{|\:\:\:if\:(B)\:S\:|\:if\:(B)\:S\:else\:S}$\\
			$\hspace{1.5cm}\mathsf{|\:\:\:while\:(B)\:S\: | \: break;}$\\
			$\hspace{1.5cm}\mathsf{|\:\:\:\{Sl\}}$\\
			$\mathsf{Sl \in \mathbb{Sl}::=Sl\:\:\:S\:|\:\backepsilon}$\\
			$\mathsf{P\in \mathbb{P}\:::=Sl}$
			}	
		\end{small}
\end{flushleft}		
\end{frame}

\begin{frame}{\rl{برچسب‌ها}}
		\begin{flushright}
\rl{
\lr{at$\llbracket\mathsf{S}\rrbracket$} : برچسب شروع $\mathsf{S}$.\\
\pause
\lr{aft$\llbracket\mathsf{S}\rrbracket$} : برچسب پایان $\mathsf{S}$، اگر پایانی داشته باشد.\\
\pause
\lr{esc$\llbracket\mathsf{S}\rrbracket$} : یک مقدار بولی را باز‌‌می‌گرداند که بسته به اینکه در $\mathsf{S}$ عبارت دستوری $\mathsf{break;}$ وجود دارد یا خیر، مقدار درست یا غلط را بر‌می‌گرداند.\\
\lr{brk-to$\llbracket\mathsf{S}\rrbracket$} : 
برچسبی است که اگر حین اجرای $\mathsf{S}$ عبارت دستوری $\mathsf{break;}$ اجرا شود، برنامه از آن برچسب ادامه پیدا می کند.\\
\lr{brks-of$\llbracket\mathsf{S}\rrbracket$} :
مجموعه‌ای از برچسب‌های عبارت‌های دستوری
$\mathsf{break;}$
که داخل $\mathsf{S}$ هستند را بر‌می‌گرداند.\\
\lr{in$\llbracket\mathsf{S}\rrbracket$} : مجموعه‌ای از تمام برچسب‌های درون $\mathsf{S}$ را برمی‌گرداند.\\
\lr{labs$\llbracket\mathsf{S}\rrbracket$} : مجموعه‌ای از تمام بر‌چسب‌هایی که با اجرای $\mathsf{S}$ قابل دسترسی هستند را بر‌می‌گرداند.\\
مجموعه‌ی همه‌ی برچسب‌ها را با 
$\mathbb{L}$
نشان می‌دهیم.
}
	\end{flushright}		
\end{frame}


\begin{frame}{\rl{معناشناسی برنامه‌ها}}
	\begin{block}{\rl{محیط }}
		\rl{
		به ازای مجموعه مقادیر
		$\mathbb{V}$
		و مجموعه متغیر‌ها
		$\mathbb{X}$ تابع 
		$\rho : \mathbb{X} \rightarrow \mathbb{V}$ 
		را یک محیط می‌گوییم. مجموعه‌ی همه‌ی محیط‌ها را با $\mathbb{EV}$ نمایش می‌دهیم.}
	\end{block}

	\begin{block}{\rl{وضعیت}}
	\rl{
		به ازای مجموعه مقادیر(وضعیت): به هر زوج مرتب متشکل از یک برچسب $l$ و یک محیط $\rho$ یک وضعیت  
		$\langle l , \rho \rangle$
		می‌گوییم. مجموعه‌ی همه‌ی وضعیت‌ها را با $\mathfrak{S}$ نشان می‌دهیم.}
\end{block}

	\begin{block}{\rl{رد پیشوندی}}
	\rl{
		به یک دنباله از وضعیت‌ها( با امکان تهی بودن) یک رد پیشوندی می‌گوییم.}
\end{block}

\end{frame}

\begin{frame}{\rl{معناشناسی}}
	\begin{block}{\rl{عملگر چسباندن}}
	\renewcommand{\baselinestretch}{0.8}
		\rl{
	برای 
	$\pi_1 , \pi_2 \in \mathfrak{S^{+\infty}}$ 
	و
	$\sigma_1 ,\sigma_2 \in \mathfrak{S}$
	داریم:
	\vspace{-0.5cm}
	\begin{center}	
	$$\blacktriangleleft \pi_1 \in \mathfrak{S}^\infty:$$
	$$\pi_1 \Join \pi_2 = \pi_1$$
	$$\blacktriangleleft \pi_1 \in \mathfrak{S}^+:$$
	$$\blacktriangleleft \blacktriangleleft \sigma_1 = \sigma_2:$$
	$$\pi_1\sigma_1 \Join \sigma_2 \pi_2 = \pi_1 \sigma_1 \pi_2$$
	$$\blacktriangleleft \blacktriangleleft \sigma_1 \neq \sigma_2:$$
	در این حالت 
	$\pi_1 \Join \pi_2$
	تعریف نمی‌شود.	
\end{center}}
	\end{block}
\end{frame}

\begin{frame}{\rl{معناشناسی}}
	\begin{block}{\rl{معنای عبارت‌های حسابی - تابع $\mathcal{A}$}}
		\renewcommand{\baselinestretch}{0.8}
		\rl{
			تابع 
			$\mathcal{A}:\mathbb{A}\rightarrow (\mathbb{EV} \rightarrow \mathbb{V})$
			را به صورت بازگشتی روی ساختار 
			$\mathsf{A} \in \mathbb{A}$
			به شکل زیر تعریف می‌کنیم:
			\vspace{-0.5cm}
			\begin{center}
			$$\blacktriangleleft\mathcal{A\llbracket\mathsf{1}\rrbracket\rho = }1     $$
			$$\blacktriangleleft\mathcal{A\llbracket\mathsf{x}\rrbracket\rho = } \rho(\mathsf{x})          $$
			$$\blacktriangleleft\mathcal{A\llbracket\mathsf{A_1-A_2}\rrbracket\rho = }\mathcal{A\llbracket\mathsf{A_1}\rrbracket\rho }- \mathcal{A\llbracket\mathsf{A_2}\rrbracket\rho }       $$
		\end{center}
		}
		
	\end{block}
\end{frame}


\begin{frame}{\rl{معناشناسی}}
	\begin{block}{\rl{معنای عبارت‌های بولی - تابع $\mathcal{B}$}}
		\renewcommand{\baselinestretch}{2}
		\rl{
		تابع 
		$\mathcal{B}: \mathbb{B} \rightarrow (\mathbb{EV} \rightarrow \mathbb{BOOL})$
		را به صورت بازگشتی روی ساختار 
		$\mathsf{B} \in \mathbb{B}$
		به شکل زیر تعریف می‌کنیم:
		\vspace{-0.25cm}
		\begin{center}
			اگر $\mathcal{A\llbracket\mathsf{A_1}\rrbracket\rho }$ کوچکتر از $\mathcal{A\llbracket\mathsf{A_2}\rrbracket\rho }$ باشد
			$\blacktriangleleft\mathcal{B\llbracket\mathsf{A_1<A_2}\rrbracket\rho = } True   \hspace{2cm}  $\\
			اگر $\mathcal{A\llbracket\mathsf{A_1}\rrbracket\rho }$ بزرگتر از $\mathcal{A\llbracket\mathsf{A_2}\rrbracket\rho }$ باشد
			$\blacktriangleleft\mathcal{B\llbracket\mathsf{A_1<A_2}\rrbracket\rho = } False   \hspace{2cm}  $\\
			$ \blacktriangleleft\mathcal{B\llbracket\mathsf{B_1 nand B_2}\rrbracket\rho = } \neg(\mathcal{B\llbracket\mathsf{B_1}\rrbracket\rho}   \wedge \mathcal{B\llbracket\mathsf{B_2}\rrbracket\rho}) $
		\end{center}
		}
	\end{block}
\end{frame}

\begin{frame}{\rl{معناشناسی}}
	\begin{block}{\rl {معنای برنامه‌ها - تابع $\mathcal{S}^*$ (دستور مقداردهی)}}
		\rl{
		\centering
		\vspace{-1cm}
		$$\blacktriangleleft\mathsf{S}=\mathsf{x\doteq A;}\;:$$
		$$\mathcal{S^*} \llbracket\mathsf{S}\rrbracket = \{ \langle at\llbracket\mathsf{S}\rrbracket , \rho \rangle | \rho \in \mathbb{EV}       \} \cup   $$  
		$$\{ \langle at\llbracket\mathsf{S}\rrbracket , \rho \rangle \langle aft\llbracket\mathsf{S}\rrbracket , \rho[\mathsf{x}\leftarrow \mathcal{A}\llbracket\mathsf{A}\rrbracket\rho] \rangle | \rho \in \mathbb{EV}       \}             $$ 
		}
	\end{block}
\end{frame}

\begin{frame}{\rl{معناشناسی}}
	\begin{block}{\rl {معنای برنامه‌ها - تابع $\mathcal{S}^*$ (دستور شرط)}}
		\rl{
			\centering
			\vspace{-1cm}
	$$\blacktriangleleft\mathsf{S}= \mathsf{if}  \mathsf{ (B) S_t}:$$
$$\mathcal{S^*} \llbracket\mathsf{S}\rrbracket = \{ \langle at\llbracket\mathsf{S}\rrbracket , \rho \rangle | \rho \in \mathbb{EV}       \} \cup     \{ \langle at\llbracket\mathsf{S}\rrbracket , \rho \rangle \langle aft\llbracket\mathsf{S}\rrbracket , \rho \rangle | \mathcal{B}\llbracket\mathsf{B}\rrbracket \rho =False      \} 
$$$$\cup    \{ \langle at\llbracket\mathsf{S}\rrbracket , \rho \rangle \langle at\llbracket\mathsf{S_t}\rrbracket , \rho \rangle 
\pi | \mathcal{B}\llbracket\mathsf{B}\rrbracket \rho =True  \wedge   \langle  at\llbracket\mathsf{S_t}\rrbracket  , \rho \rangle \pi \in \mathcal{S} \llbracket\mathsf{S_t}\rrbracket    \}          $$ 
		}
	\end{block}
\end{frame}


\begin{frame}{\rl{معناشناسی}}
	\begin{block}{\rl {معنای برنامه‌ها - تابع $\mathcal{S}^*$ (دستور حلقه)}}
		\rl{
			\centering
			\vspace{-1cm}
			$$\mathcal{S^*} \llbracket\mathsf{S}\rrbracket = lfp^{\subseteq}\: \mathcal{F\llbracket\mathsf{S}\rrbracket},$$
			$$\mathcal{F} \llbracket\mathsf{S}\rrbracket X= \{ \langle at\llbracket\mathsf{S}\rrbracket , \rho \rangle | \rho \in \mathbb{EV}       \} \cup $$
			$$  \{ \pi_2 \langle l ,\rho \rangle \langle aft\llbracket\mathsf{S}\rrbracket,\rho \rangle |  \pi_2 \langle l ,\rho \rangle \in X \wedge \mathcal{B}\llbracket\mathsf{B}\rrbracket\rho=False \wedge l= at\llbracket\mathsf{S}\rrbracket   \} \cup      $$
			$$  \{ \pi_2 \langle l ,\rho \rangle \langle at\llbracket\mathsf{S_b}\rrbracket,\rho \rangle \pi_3 |  \pi_2 \langle l ,\rho \rangle \in X \wedge \mathcal{B}\llbracket\mathsf{B}\rrbracket\rho=True \wedge$$$$  \langle at\llbracket\mathsf{S_b}\rrbracket,\rho \rangle \pi_3 \in  \mathcal{S} \llbracket\mathsf{S_b}\rrbracket   \wedge   l= at\llbracket\mathsf{S}\rrbracket  \}  $$
		}
	\end{block}
\end{frame}

%----------------------------------------
\section{\rl{صوری‌سازی جدید برای روش وارسی مدل}}

\begin{frame}{\rl{عبارات منظم}}
\begin{block}{\rl{نحو عبارات منظم}}
	\rl{مجموعه‌ی $\mathbb{R}$ توسط گرامر زیر ساخته می‌شود.}\\
	\begin{flushright}
	\setlength{\parindent}{3.5cm}
	\begin{small}
	\lr{
	\hspace{2cm}$\mathsf{R} ::=\hspace{0.5cm} \varepsilon$\\
	$|\:\:\:\mathsf{L : B}$\\
	$|\:\:\:\mathsf{R_1 R_2}\:\:\:(or\:\mathsf{R_1 \bullet R_2 })$\\
	$|\:\:\:\mathsf{R_1\:\mid\:R_2}\:\:\:(or\:\mathsf{R_1 + R_2 })$\\
	$|\:\:\:\mathsf{R_1^*}$\\
	$|\:\:\:\mathsf{R_1^+}$\\
	$|\:\:\:(\mathsf{R_1})$
	}
	\end{small}
	\end{flushright}
\end{block}
\end{frame}

\begin{frame}{\rl{عبارات منظم}}
	\begin{block}{\rl{معناشناسی عبارات منظم}}
		\centering
		\vspace{-1cm}
$$\blacktriangleleft\mathcal{S}^r\llbracket\varepsilon\rrbracket= \{ \langle \underline{\rho} , \epsilon \rangle | \underline{\rho} \in \underline{\mathbb{EV}}\}$$
\pause
$$\blacktriangleleft\mathcal{S}^r\llbracket\mathsf{L:B}\rrbracket = \{\langle\underline{\rho},\langle l , \rho \rangle \rangle | l \in \mathsf{L} \land \mathcal{B}\llbracket\mathsf{B}\rrbracket \underline{\rho},\rho \}$$
\pause
$$\blacktriangleleft\mathcal{S}^r\llbracket\mathsf{R_1 R_2}\rrbracket= \mathcal{S}^r\llbracket\mathsf{R_1}\rrbracket \Join \mathcal{S}^r\llbracket\mathsf{R_2}\rrbracket$$
\pause
$$\blacktriangleleft\mathcal{S}^r \llbracket\mathsf{R_1\:|\:R_2}\rrbracket= 
\mathcal{S}^r \llbracket R_1\rrbracket \cup
\mathcal{S}^r \llbracket R_2\rrbracket$$
	\end{block}
\end{frame}


\begin{frame}{\rl{عبارات منظم}}
	\begin{block}{\rl{معناشناسی عبارات منظم (ادامه)}}
		\centering
		\vspace{-1cm}
		$$\blacktriangleleft\mathcal{S}^r \llbracket\mathsf{R}\rrbracket^0 = \mathcal{S}^r\llbracket\varepsilon\rrbracket,$$
		$$\blacktriangleleft\mathcal{S}^r \llbracket\mathsf{R}\rrbracket^{n+1} = \mathcal{S}^r \llbracket\mathsf{R}\rrbracket^{n} \Join
		\mathcal{S}^r \llbracket\mathsf{R}\rrbracket.$$
		\pause
$$\blacktriangleleft\mathcal{S}^r\llbracket\mathsf{R^*}\rrbracket =  \bigcup_{n \in \mathbb{N}}
\mathcal{S}^r \llbracket\mathsf{R}^n\rrbracket,$$
		\pause
$$\blacktriangleleft\mathcal{S}^r\llbracket\mathsf{R^+}\rrbracket =  \bigcup_{n \in \mathbb{N}\setminus\{0\}}
\mathcal{S}^r \llbracket\mathsf{R}^n\rrbracket.$$
		\pause
$$\blacktriangleleft\mathcal{S}^r \llbracket(\mathsf{R})\rrbracket=\mathcal{S}^r \llbracket\mathsf{R}\rrbracket.$$
	\end{block}
\end{frame}

\begin{frame}{\rl{عبارات منظم}}
		\begin{block}{\rl{ارضا پذیری}}
		\rl{
			می‌گوییم، در محیط اولیه‌ی $\underline{\rho}$ رد پیشوندی $\pi$ عبارت منظم $\mathsf{R}$ را ارضا می‌کند، اگر و تنها اگر 
		$\langle \underline{\rho},\pi \rangle \in \mathcal{S}^r \llbracket \mathsf{R} \rrbracket$. 	
		}
	\end{block}
\end{frame}



\begin{frame}{\rl{گونه‌های مختلف عبارات منظم}}
	\begin{block}{\rl{عبارت منظم تهی - $\mathbb{R_\varepsilon}$}}
		\rl{
			$\mathbb{R}_\varepsilon$ 
			توسط گرامر زیر تولید می‌شود.
			\vspace{-0.75cm}
			\begin{center}
			$$\mathsf{R}\:\:\: ::= \:\:\: \varepsilon \: | \: \mathsf{R_1 R_2} \: | 
			\: \mathsf{R_1 + R_2} \: | \: \mathsf{R_1^*} \: | \: \mathsf{R_1^+} | (\mathsf{R_1})$$
			\end{center}
		}
	\end{block}
\end{frame}

	\begin{frame}{\rl{گونه‌های مختلف عبارات منظم}}
		\begin{block}{\rl{عبارات منظم ناتهی - $\mathbb{R}^+$}}
			\rl{
$\mathbb{R}^+$ توسط گرامر زیر تولید می‌شود.	توسط گرامر زیر تولید می‌شود.
				\vspace{-0.75cm}
				\begin{center}
					$$\mathsf{R}\:\:\: ::= \:\:\: \mathsf{L:B} \: |
					\: \mathsf{\varepsilon R_2} \: | \: \mathsf{R_1 \varepsilon} \: |
					\: \mathsf{R_1 R_2} \: | 
					\: \mathsf{R_1 + R_2} \: | \: \mathsf{R_1^+} | (\mathsf{R_1})$$
				\end{center}
			}
		\end{block}
	\end{frame}


	\begin{frame}{\rl{گونه‌های مختلف عبارات منظم}}
	\begin{block}{\rl{عبارات منظم بدون انتخاب - $\mathbb{R}^\nmid$}}
		\rl{
$\mathbb{R}^\nmid$ توسط گرامر زیر تولید می‌شود.
			\vspace{-0.75cm}
			\begin{center}
$$
\mathsf{R}\:\:\: ::= \:\:\: \varepsilon \: | \: \mathsf{L:B} \: | \:
\mathsf{R_1 R_2} \: | \: \mathsf{R_1}^* \: | \: \mathsf{R_1}^+ \: | \:
(\mathsf{R_1})$$
			\end{center}
		}
	\end{block}
\end{frame}


\begin{frame}{\rl{صورت جدید مسئله‌ی وارسی مدل}}
	\begin{block}{\rl{بستار پیشوندی}}
		\rl{
	اگر 
$\Pi \in \mathit{P}(\mathbb{\underline{EV}\times \mathfrak{S}^+})$،
آنگاه بستار پیشوندی $\Pi$ را به صورت زیر تعریف می‌کنیم:
			\vspace{-0.75cm}
			\begin{center}
$$\mathsf{prefix}(\Pi)=
\{\langle \underline{\rho},\pi\rangle | \pi \in \mathfrak{S^+} \land \exists\
\pi' \in \mathfrak{S^*}: \langle \underline{\rho}, \pi \pi' \rangle \in \Pi \}$$
			\end{center}
		}
	\end{block}
\end{frame}


\begin{frame}{\rl{صورت جدید مسئله‌ی وارسی مدل}}
	\begin{block}{\rl{وارسی مدل}}
		\rl{
		اگر 
		$\mathsf{P}\in\mathbb{P} , \mathsf{R} \in \mathbb{R}^+ , \underline{\rho} \in \underline{\mathbb{EV}}$
		آنگاه:
			\vspace{-0.75cm}
			\begin{center}
$$\mathsf{P},\underline{\rho} \models \mathsf{R}
\Leftrightarrow
(\{\underline{\rho}\}\times \mathcal{S}^* \llbracket\mathsf{P}\rrbracket) \subseteq 
\mathsf{prefix} (\mathcal{S}^r \llbracket\mathsf{R} \bullet (?:\mathit{T})^*\rrbracket)
$$
			\end{center}
		}
	\end{block}
\end{frame}

\begin{frame}{\rl{در مورد توقف پذیری}}
	\begin{block}{\rl{توقف پذیری}}
		\rl{
			برنامه‌ی $\mathsf{P}$ را به همراه محیط اولیه
			$\underline{\rho}$
			توقف پذیر
			می‌گوییم، اگر و تنها اگر وجود داشته باشد 
			$\pi \in \mathcal{S}^* \llbracket \mathsf{P} \rrbracket$، 
			که ($\rho$ محیط متناظر با محیط اولیه‌ی $\underline{\rho}$ است):
			$$\pi = \langle at \llbracket \mathsf{P} \rrbracket , \rho \rangle \pi'$$
			و اینکه $\langle aft\llbracket \mathsf{P} \rrbracket , \rho' \rangle$ در $\pi$ حضور داشته باشد. در این صورت می‌نویسیم
			$\mathsf{P,\underline{\rho}\downarrow}$.
		}
	\end{block}
\end{frame}

\begin{frame}{\rl{در مورد توقف پذیری}}
\begin{exampleblock}{\rl{قضیه}}
	\rl{
برای برنامه‌ی $\mathsf{P}$ و محیط اولیه‌ی $\underline{\rho}$ داریم $\mathsf{P} , \underline{\rho} \downarrow $، اگر و تنها اگر $\rho$ محیط متناظر با محیط اولیه‌ی $\underline{\rho}$ باشد و
\vspace{-1cm}
\begin{center}
$$\forall \pi \in \mathfrak{S^{+}} : 
\langle at \llbracket \mathsf{P} \rrbracket , \rho \rangle \pi \in \mathcal{S^*} \llbracket \mathsf{P} \rrbracket \rightarrow
\langle at \llbracket \mathsf{P} \rrbracket , \rho \rangle \pi \in \mathbb{R^+}.$$	
\end{center}
}
\end{exampleblock}
\end{frame}

\begin{frame}{\rl{در مورد توقف پذیری}}
	\vspace{-1cm}
	\begin{flushright}
	\rl{$\bullet$
		داریم:}
\end{flushright}
	\vspace{-1.5cm}
	\begin{center}
	$$\mathsf{P},\underline{\rho} \models \varepsilon$$
	$$\Leftrightarrow$$
	$$(\{\underline{\rho}\}\times \mathcal{S}^* \llbracket \mathsf{P}\rrbracket) \subseteq 
	\mathsf{prefix} (\mathcal{S}^r \llbracket \varepsilon \bullet (?:\mathit{T})^*\rrbracket)
	=\mathsf{prefix} (\mathcal{S}^r \llbracket (?:\mathit{T})^*\rrbracket)$$
\end{center}
\end{frame}


\begin{frame}{\rl{در مورد توقف پذیری}}
	\vspace{-1cm}
	\begin{flushright}
		\rl{$\bullet$
			پس اگر الگوریتمی برای تشخیص $\mathsf{P},\underline{\rho} \models \varepsilon$
	داشته باشیم، مسئله‌ی توقف حل می‌شود.\\
	\pause
	\vspace{1 cm}
	$\bullet$
	پس پیاده‌سازی این روش غیر ممکن است.\\
	\pause
	\vspace{1 cm}
	$\bullet$
	دو صورت دیگر هم قابل پیاده‌سازی نیستند!
	}
	\end{flushright}
\end{frame}
%----------------------------------
\begin{frame}{\rl{وارسی مدل منظم}}
	\begin{flushright}
		\rl{$\bullet$
			ساختار عبارات منظم به صورت اضافه می‌شود.}
	\end{flushright}
\end{frame}

\begin{frame}{\rl{برخی مفاهیم در مورد عبارات منظم}}
	\begin{block}{\rl{هم‌ارزی عبارات منظم}}
		\rl{
			دو عبارت منظم
			$\mathsf{R}_1$
			و
			$\mathsf{R_2}$
			را هم‌ارز می‌گوییم، اگر و تنها اگر شرط زیر برقرار باشد:
			\vspace{-1.25cm}
			\begin{center}
				$$\mathcal{S}^r \llbracket \mathsf{R_1} \rrbracket = \mathcal{S}^r \llbracket \mathsf{R_2} \rrbracket$$
			\end{center} 
			و آن را با 
			$\mathsf{R_1} \Bumpeq \mathsf{R_2}$
			نمایش می‌دهیم.
		}
	\end{block}

\begin{exampleblock}{\rl{قضیه}}
	\rl{
	هم‌ارزی $\Bumpeq$ تعریف شده روی مجموعه‌ی عبارت‌های منظم یک رابطه‌ی هم‌ارزی است.
	}
\end{exampleblock}
\end{frame}

\begin{frame}{\rl{برخی مفاهیم در مورد عبارات منظم}}
	\begin{block}{\rl{تابع $\mathsf{dnf}$}}
		\rl{
		تابع $\mathsf{dnf}$ روی عبارات منظم به شکل زیر تعریف می‌شود:
		\vspace{-1cm}
		\begin{center}
		$$\blacktriangleleft\mathsf{dnf}(\varepsilon)=\varepsilon$$
		\pause
		$$\blacktriangleleft\mathsf{dnf}(\mathsf{L:B})=\mathsf{L:B}$$
		\pause
		$$\blacktriangleleft\mathsf{dnf}(\mathsf{R_1 R_2})= \mathsf{\Sigma_{i=1}^{n_1} \Sigma_{j=1}^{n_2} R_1^i R_2^j }$$
		$$\mathsf{where\;R_1^1 + R_1^2 + ... + R_1^{n_1} = dnf(R_1)\;and\; R_2^1 + R_2^2 + ... + R_2^{n_2}= dnf(R_2)}$$
		\end{center}
		}
	\end{block}
\end{frame}

\begin{frame}{\rl{برخی مفاهیم در مورد عبارات منظم}}
	\begin{block}{\rl{
				تابع 
				$\mathsf{dnf}$
			(ادامه)}}
		\rl{
			\vspace{-1cm}
			\begin{center}
   $$\blacktriangleleft\mathsf{dnf (R_1+R_2)=dnf(R_1)+dnf(R_2)}$$
   \pause
$$\blacktriangleleft\mathsf{dnf (R^*)}= \mathsf{((R_1)^* (R_2)^* ... (R_n)^*)^*}$$
\pause
$$\blacktriangleleft\mathsf{dnf(R^+)=dnf(RR^*)}$$
\pause
   $$\blacktriangleleft\mathsf{ dnf((R)) = ( dnf(R) ) }$$  
			\end{center}
		}
	\end{block}
\end{frame}

\begin{frame}{\rl{برخی مفاهیم در مورد عبارات منظم}}
	\begin{block}{\rl{
				تابع 
				$\mathsf{dnf}$
				(ادامه)}}
		\rl{
			\vspace{-1.5cm}
			\begin{center}
   $$\blacktriangleleft\mathsf{dnf (R^*)}= \mathsf{((R_1)^* (R_2)^* ... (R_n)^*)^*}$$
$$\mathsf{where\;dnf(R)=R_1+R_2+...+R_n}$$
			\end{center}
		}
	\end{block}
\pause
\begin{exampleblock}{\rl{قضیه}}
	\rl{
	اگر $\mathsf{R} \in \mathbb{R}$ آنگاه $\mathsf{dnf(R)}$ یک ترکیب نرمال فصلی است و داریم:
	\begin{center}
		\vspace{-1cm}
			$$\mathsf{dnf(R) \Bumpeq R}$$
	\end{center}
	}
\end{exampleblock}
\end{frame}

\begin{frame}{\rl{برخی مفاهیم در مورد عبارات منظم}}
	\begin{block}{\rl{
				تابع 
				$\mathsf{fstnxt}$}}
		\rl{
			\vspace{-1cm}
			\begin{center}
				$
				\blacktriangleleft\mathsf{fstnxt (L:B) = \langle L:B , \varepsilon \rangle }
				$\\
				\pause
				$\blacktriangleleft\mathsf{fstnxt (R_1 R_2) = fstnxt (R_2)}\;\;$\\
				$\mathsf{where\; R_1 \in \mathbb{R}_\varepsilon}$\\
				\pause
				$\blacktriangleleft\mathsf{fstnxt(R_1 R_2)= \llparenthesis  R_1^n \in \mathbb{R}_\varepsilon \; \mathbb{?} \; \langle R_1^f , R_2 \rangle : 
					\langle R_1^f , R_1^n \bullet R_{2} \rangle \rrparenthesis} $\\
				$\mathsf{where\; R_1 \notin \mathbb{R}_\varepsilon and\;
					fstnxt(R_1)= \langle R_1^f , R_1^n \rangle}$\\
			\end{center}
		}
	\end{block}
\end{frame}

\begin{frame}{\rl{برخی مفاهیم در مورد عبارات منظم}}
	\vspace{-1cm}
	\begin{block}{\rl{
				تابع 
				$\mathsf{fstnxt}$
			(ادامه)}}
		\rl{
			\vspace{-1cm}
			\begin{center}
				$\blacktriangleleft\mathsf{fstnxt(R^+)= \llparenthesis R^n \in \mathbb{R_\varepsilon}\;?\;\langle R^f , R^* \rangle : \langle R^f , R^n \bullet R^* \rangle \rrparenthesis}\;\;$\\
				\pause
				$\mathsf{where\; fstnxt(R)=\langle R^f , R^n \rangle}$\\
				$\blacktriangleleft\mathsf{fstnxt((R))=fstnxt(R)}$
			\end{center}
		}
	\end{block}
\pause
	\begin{exampleblock}{\rl{قضیه}}
		\rl{
				برای هر عبارت منظم 
		$\mathsf{R} \in \mathbb{R^+ \cap R^\nmid}$،
		اگر 
		$\mathsf{fstnxt(R)=\langle L:B , R'\rangle}$
		آنگاه 
		$\mathsf{R'}\in \mathbb{R}^\nmid$
		و 
		$\mathsf{R \Bumpeq L:B\bullet R'}$
	}
	\end{exampleblock}

\end{frame}


\begin{frame}{\rl{صورت وارسی مدل منظم}}
	\begin{block}{\rl{وارسی‌گر رد پیشوندی- تابع $\mathcal{M}^t$}}
		\rl{	\vspace{-1cm}	
	\begin{center}
$$\blacktriangleleft\mathcal{M}^t \langle \underline{\rho} , \varepsilon \rangle \pi = 
\langle \mathit{T} , \varepsilon \rangle$$
\pause
$$\blacktriangleleft\mathcal{M}^t \langle \underline{\rho} , \mathsf{R} \rangle \epsilon = 
\langle \mathit{T} , \mathsf{R} \rangle$$
\pause
$$\blacktriangleleft\mathcal{M}^t \langle \underline{\rho} , \mathsf{R} \rangle \pi = 
\llparenthesis \langle \underline{\rho}, \langle l_1, \rho_1 \rangle \rangle \in
\mathcal{S}^r \llbracket \mathsf{L:B} \rrbracket \; ? \; 
\mathcal{M}^t \langle \underline{\rho}, \mathsf{R'} \rangle \pi' \; : \; 
\langle \mathit{F}, \mathsf{R} \rangle \rrparenthesis$$
$$\mathsf{where}\; \pi=\langle l_1,\rho_1\rangle \pi' \;\mathsf{and}\; 
\langle \mathsf{L:B,R'}\rangle = \mathsf{fstnxt(R)}$$	
\end{center}
}
	\end{block}
\end{frame}



\begin{frame}{\rl{صورت وارسی مدل منظم}}
	\begin{block}{\rl{وارسی مدل منظم محدود به $\mathbb{R^\nmid}$- تابع $\mathcal{M}^\nmid$}}
		\rl{	\vspace{-1cm}	
			\begin{center}
					 $$\mathcal{M}^\nmid \langle \underline{\rho} , \mathsf{R} \rangle \Pi= \{ \langle \pi , \mathsf{R'} \rangle | \pi \in \Pi \land  
				\mathcal{M}^t \langle \underline{\rho} , \mathsf{R} \rangle \pi =
				\langle \mathit{T} , \mathsf{R'} \rangle \}$$
			\end{center}
		}
	\end{block}
\end{frame}




\begin{frame}{\rl{صورت وارسی مدل منظم}}
	\begin{block}{\rl{وارسی مدل منظم- تابع $\mathcal{M}$}}
		\rl{	\vspace{-1cm}	
			\begin{center}
				$$\mathcal{M} \langle \underline{\rho} , \mathsf{R} \rangle \Pi= \bigcup_{i=1}^n \{  \langle \underline{\rho},\pi\rangle | \exists \mathsf{R'} \in \mathbb{R}:\; \langle \pi , \mathsf{R'} \rangle \in 
				\mathcal{M}^\nmid \langle \underline{\rho}, \mathsf{R_i} \rangle \Pi \}$$
				$$\mathsf{where \; dnf (R) = R_1 + R_2 + ... + R_n}$$ 
			\end{center}
		}
	\end{block}
\end{frame}


\begin{frame}{\rl{صورت وارسی مدل منظم}}
	\begin{block}{\rl{وارسی مدل منظم}}
		\rl{	\vspace{-1cm}	
			\begin{center}
اگر ویژگی $\mathsf{R} \in \mathbb{R}$ در محیط اولیه‌ی $\underline{\rho}$ برای برنامه‌ی $\mathsf{P}$ برقرار باشد، می نویسیم
$$\mathsf{P} , \underline{\rho} \models_r \mathsf{R}$$
و برقرار بودن این رابطه با شرط زیر تعریف می‌شود:
$$\mathsf{P} , \underline{\rho} \models_r \mathsf{R} \iff \{\underline{\rho}\} \times \mathcal{S}^* \llbracket \mathsf{P} \rrbracket \subseteq \mathcal{M} \langle \underline{\rho} , \mathsf{R} \rangle \mathcal{S}^* \llbracket \mathsf{P} \rrbracket$$
			\end{center}
		}
	\end{block}
\end{frame}


\begin{frame}{\rl{درستی و تمامیت وارسی مدل منظم}}
	\begin{exampleblock}{\rl{قضیه درستی و تمامیت}}
		\rl{	\vspace{-1cm}
			اگر $\mathsf{P}$ یک برنامه، $\mathsf{R}$ یک عبارت منظم و $\underline{\rho}$ یک محیط اولیه باشند، آنگاه داریم:	
			\begin{center}
				$$\mathsf{P},\underline{\rho} \models_r \mathsf{R} \iff
				\mathsf{P},\underline{\rho} \models \mathsf{R}$$
			\end{center}
		}
	\end{exampleblock}
\end{frame}
%------------------------------------------
\section{\rl{وارسی مدل ساختارمند}}
\begin{frame}{\rl{وارسی مدل ساختارمند}}
	\begin{flushright}
		\rl{$\bullet$
			ساختار برنامه‌ها به صورت اضافه می‌شود.}
	\end{flushright}
\end{frame}


\begin{frame}{\rl{صورت وارسی مدل ساختارمند}}
	\begin{flushright}
		\rl{
			\vspace{-0.75cm}	
		\begin{block}{\rl{وارسی مدل ساختارمند- تابع 
					$\mathcal{\hat{M}}$
				(دستور مقداردهی)}}
		\rl{	\vspace{-0.5cm}	
			\begin{footnotesize}
			برای $\mathsf{S=\;x \doteq A;}$ و $\mathsf{R} \in \mathbb{R^\nmid} \cap \mathbb{R^+}$ داریم ($\mathsf{where \; fstnxt(R)=\langle L:B,R' \rangle}$):
			\end{footnotesize}
		\vspace{-0.5 cm}
			\begin{center}
				\begin{footnotesize}
					\pause
					$$\blacktriangleleft\mathcal{\hat{M}^\nmid} \langle \underline{\rho}, \mathsf{R} \rangle \llbracket \mathsf{S} \rrbracket
					=$$
					\pause
					$$\{
					\langle \langle at \llbracket \mathsf{S} \rrbracket, \rho \rangle , \mathsf{R'} \rangle |
					\langle \underline{\rho}, \langle at \llbracket \mathsf{S} \rrbracket, \rho \rangle \rangle \in \mathcal{S}^r \llbracket \mathsf{L:B} \rrbracket \}$$
					\pause
					$$\cup \{ \langle \langle at \llbracket \mathsf{S} \rrbracket , \rho \rangle  \langle aft \llbracket \mathsf{S} \rrbracket, 
					\rho [\mathsf{x}\leftarrow \mathcal{A}\llbracket \mathsf{A} \rrbracket \rho ]\rangle , \varepsilon \rangle | \mathsf{R'} \in \mathbb{R_\varepsilon} \land
					\langle \underline{\rho}, \langle at \llbracket \mathsf{S} \rrbracket , \rho \rangle \rangle \in 
					\mathcal{S}^r \llbracket \mathsf{L:B} \rrbracket
					\}$$
					\pause
					$$\cup \{
					\langle \langle at \llbracket \mathsf{S} \rrbracket , \rho \rangle \langle aft \llbracket \mathsf{S} \rrbracket , \rho [\mathsf{x} \leftarrow \mathcal \llbracket \mathsf{A}\rrbracket \rho]\rangle, \mathsf{R''}\rangle | \mathsf{R'} \notin \mathbb{R_\varepsilon} \land$$
					$$\langle \underline{\rho},\langle at \llbracket \mathsf{S} \rrbracket , \rho \rangle \rangle \in \mathcal{S}^r \llbracket \mathsf{L:B} \rrbracket \land \langle \mathsf{L':B',R''} \rangle = 
					\mathsf{fstnxt(R') }\land$$
					$$\langle \underline{\rho}, \langle aft \llbracket \mathsf{S} \rrbracket,\rho 
					[\mathsf{x} \leftarrow \mathcal{A}\llbracket \mathsf{A} \rrbracket \rho]\rangle \rangle \in 
					\mathcal{S}^r \llbracket \mathsf{L':B'} \rrbracket
					\}$$
				\end{footnotesize}
			\end{center}	
	}
\end{block}	
	}
	\end{flushright}
\end{frame}





\begin{frame}{\rl{درستی و تمامیت وارسی مدل ساختارمند}}
	\begin{exampleblock}{\rl{قضیه درستی و تمامیت}}
		\rl{	\vspace{-1cm}
			اگر $\mathsf{P}$ یک برنامه، $\mathsf{R}$ یک عبارت منظم و $\underline{\rho}$ یک محیط اولیه باشند، آنگاه داریم:	
			\begin{center}
				$$\mathsf{P},\underline{\rho} \models_s \mathsf{R} \iff
				\mathsf{P},\underline{\rho} \models_r \mathsf{R}$$
			\end{center}
		\pause
		به این شکل که ثابت می‌شود( قوی‌تر از عبارت بالا):
		\vspace{-0.75cm}
		\begin{center}
			$$\mathcal{\hat{M}} \langle \underline{\rho}, \mathsf{R} \rangle \llbracket \mathsf{P} \rrbracket
			=
			\mathcal{M} \langle \underline{\rho}, \mathsf{R} \rangle \llbracket \mathsf{P} \rrbracket
			$$
		\end{center}
		}
	\end{exampleblock}
\end{frame}

%-----------------------------------------


\section{\rl{جمع بندی و ادامه‌ی راه}}

\begin{frame}{\rl{نتیجه گیری و ادامه‌ی راه}}
\rl{
\begin{flushright}	
$\bullet$	
وارسی مدل در صورت جدیدی بیان شد.\\
\pause
$\bullet$	
قابل پیاده‌سازی نیست.\\
\pause
$\bullet$	
متناهی در نظر گرفتن وضعیت‌ها\\
\pause
$\bullet$	
حذف دو عملگر $^*$ و $^+$ از عبارات منظم
\end{flushright}
}	
\end{frame}



%--------------------------------------
\begin{frame}
	\begin{center}
		\begin{Huge}
		\rl{پایان}
	\end{Huge}
	\end{center}
\end{frame}








\end{document}
