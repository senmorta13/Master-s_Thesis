% !TeX root = UT-Thesis-Template.tex
\chapter{وارسی مدل ساختارمند}

در این فصل همان طور که پیشتر هم بارها اشاره کردیم قرار است، به ادامه‌ی ساختارمندتر کردن کار بپردازیم. در فصل گذشته ساختار عبارات منظم را به تعریف وارسی مدل اضافه کردیم و حالا می‌خواهیم ساختار زبانمان را به کار اضافه کنیم. این آخرین تلاش \cite{calcul} برای گسترش کار بوده. یعنی وارسی مدل به شکل جدید تعریف شده و معادل بودن آن با صورت قبلی وارسی مدل ثابت شده و پس از آن کار پایان می‌پذیرد. 
تعریف صورت جدید چونکه روی ساختار زبان انجام گرفته جزئیات بسیار طولانی‌ای دارد. همی باعث شده تا اثبات‌ برابری این صورت با صورت قبلی هم بسیار مفصل و حجیم باشد. این اثبات در \cite{calcul} به طور کامل حین معرفی هر مورد تعریف بیان شده. بنابراین از ارائه‌ی دوباره‌ی این جزئیات به طور کامل خودداری کرده‌ایم و صرفا یک گزارش کلی از اینکه هر برقراری چگونه ثابت شده، به سبک \cite{calcul} بعد از بیان هر مورد از تعریف، آورده‌ایم.

تعریف روی ساختار مجموعه‌ی 
$\mathbb{P \cup Sl \cup S}$
انجام شده. یعنی در واقع روی همه‌ی اعضای تعریف هر یک از این سه بخش زبان تعریف کردن صورت گرفته، تقریبا کاری شبیه به اثبات لمی که در بخش 2.5 داشتیم.

\begin{defn}
	
تابع
$\mathcal{\hat{M}}$
 را از نوع
 $\mathbb{(\underline{EV} \times R)} \rightarrow  \mathit{P}({\mathfrak{S}^{+\infty})}
\rightarrow \mathit{P}(\mathfrak{S}^{+\infty} ) $
وارسی مدل ساختارمند می‌نامیم.
\end{defn}